\documentclass{beamer}
\usetheme{Boadilla}
\usepackage{booktabs}
\usepackage{adjustbox}
\usefonttheme{serif}
\subtitle{Using Beamer}
\usepackage{pgffor}
\usepackage{graphicx}
\usepackage{booktabs}




\title{ \textbf{Stepscan Project}}
\subtitle{Results of Worksheet 1-3}
\date{\today}
\author{Saeed Kazemi}
\institute{ University of New Brunswick}
\usepackage{caption}

\begin{document}


%%%%%%%%%%%%%%%%%%%%%%%%%%%%%%%%%%%%%%%%%%%
%%%%%%%%%%%%%%%%%%%%%%%%%%%%%%%%%%%%%%%%%%%
\begin{frame}
\titlepage
\end{frame}


%%%%%%%%%%%%%%%%%%%%%%%%%%%%%%%%%%%%%%%%%%%
%%%%%%%%%%%%%%%%%%%%%%%%%%%%%%%%%%%%%%%%%%%
\begin{frame}
\frametitle{Outline}
\tableofcontents
\end{frame}


%%%%%%%%%%%%%%%%%%%%%%%%%%%%%%%%%%%%%%%%%%%
%%%%%%%%%%%%%%%%%%%%%%%%%%%%%%%%%%%%%%%%%%%
\iffalse

\section{Test Size}
    \begin{frame}
    \frametitle{Comparing different test size over Pressure Features}
    \tiny
    \begin{table}
    \centering
    \captionsetup{labelformat=empty}
    \caption{\footnotesize Comparing Accuracy and EER of different test size over Pressure Features. These results are average of both sides.}
    \input{Manuscripts/src/tables/testsize}
    \end{table}
    
    \begin{block}{\footnotesize Conditions}
        \tiny These results are average over the results of below conditions:
        \begin{itemize} 
            \item min, mean and median criteria,
            \item Both correlation and distance score matrix
            \item Keeping 95\% variances
            \item Z-score algorithm
        \end{itemize} 
    \end{block}
    
    \end{frame}
    
    
    
    \begin{frame}
    \centering
    \frametitle{Comparing different test size over Pressure Features}
    \includegraphics[scale=0.3]{Manuscripts/src/figures/testsize.png}
    \end{frame}

    
    

\fi

\section{Score Matrix}
    \foreach \n in {pfeatures}{%{afeatures-simple, afeatures-otsu, pfeatures, COAs-otsu, COAs-simple, COPs}{
    \begin{frame}
    \frametitle{Euclidean distance vs Correlation over \n}
    \tiny
    \begin{table}
    \centering
    \captionsetup{labelformat=empty}
    \caption{\footnotesize The accuracy of Euclidean distance and Correlation on \n.}
    \input{Manuscripts/src/tables/\n-Mode-Acc}
    \end{table}
    \begin{table}
    \centering
    \captionsetup{labelformat=empty}
    \caption{\footnotesize The EER of Euclidean distance and Correlation on \n}
    \label{tab:parameters condition}
    \input{Manuscripts/src/tables/\n-Mode-EER}
    \end{table}
    
    \begin{block}{\footnotesize Conditions}
        \tiny These results are average over the results of below conditions: min, mean and median criteria, All PCs and 95\% variances, Min-Max and z-score algorithm. also all features in \n \ were considered.
    \end{block}
    
    \end{frame}
    
    
    
    \begin{frame}
    \centering
    \frametitle{Euclidean distance vs Correlation (ROC curve) over \n}
    \includegraphics[scale=0.3]{Manuscripts/src/figures/\n-Mode.png}
    \end{frame}
    
    }

%%%%%%%%%%%%%%%%%%%%%%%%%%%%%%%%%%%%%%%%%%%
%%%%%%%%%%%%%%%%%%%%%%%%%%%%%%%%%%%%%%%%%%%
\section{PCA}
    \foreach \n in {pfeatures}{%{afeatures-simple, afeatures-otsu, pfeatures, COAs-otsu, COAs-simple, COPs}{
    \begin{frame}
    \frametitle{All PCs vs 95\% variance over \n}
    \tiny
    \begin{table}
    \centering
    \captionsetup{labelformat=empty}
    \caption{\footnotesize The accuracy of All PCs and 95\% variance over \n}
    \input{Manuscripts/src/tables/\n-PCA-Acc}
    \end{table}
    \begin{table}
    \centering
    \captionsetup{labelformat=empty}
    \caption{\footnotesize The EER of All PCs and 95\% variance over \n}
    \label{tab:parameters condition}
    \input{Manuscripts/src/tables/\n-PCA-EER}
    \end{table}
    
    \begin{block}{\footnotesize Conditions}
        \tiny These results are average over the results of below conditions: min, mean and median criteria, correlation and distant score, Min-Max and z-score algorithm. also all features in \n \ were considered.
    \end{block}
    
    \end{frame}
    
    
    
    \begin{frame}
    \centering
    \frametitle{All PCs vs 95\% variance over \n}
    \includegraphics[scale=0.3]{Manuscripts/src/figures/\n-PCA.png}
    \end{frame}
    
    }
\section{Normalization algorithm}
    \foreach \n in {pfeatures}{%{afeatures-simple, afeatures-otsu, pfeatures, COAs-otsu, COAs-simple, COPs}{
    \begin{frame}
    \frametitle{Min-Max vs z-score over \n}
    \tiny
    \begin{table}
    \centering
    \captionsetup{labelformat=empty}
    \caption{\footnotesize The accuracy of Min-Max and z-score algorithm over \n}
    \input{Manuscripts/src/tables/\n-Normalizition-Acc}
    \end{table}
    \begin{table}
    \centering
    \captionsetup{labelformat=empty}
    \caption{\footnotesize The EER of Min-Max and z-score algorithm over \n}
    \label{tab:parameters condition}
    \input{Manuscripts/src/tables/\n-Normalizition-EER}
    \end{table}
    
    \begin{block}{\footnotesize Conditions}
        \tiny These results are average over the results of below conditions: min, mean and median criteria, correlation and distant score, All PCs and 95\% variances. also all features in \n \ were considered.
    \end{block}
    
    \end{frame}
    
    
    
    \begin{frame}
    \centering
    \frametitle{Min-Max vs z-score over \n}
    \includegraphics[scale=0.3]{Manuscripts/src/figures/\n-Normalizition.png}
    \end{frame}
    
    }
\section{Criteria}
    \foreach \n in {pfeatures}{%{afeatures-simple, afeatures-otsu, pfeatures, COAs-otsu, COAs-simple, COPs}{
    \begin{frame}
    \frametitle{Comparing different criteria over \n}
    \tiny
    \begin{table}
    \centering
    \captionsetup{labelformat=empty}
    \caption{\footnotesize The accuracy of The EER of min, mean and median criteria over \n}
    \input{Manuscripts/src/tables/\n-Model-Type-Acc}
    \end{table}
    \begin{table}
    \centering
    \captionsetup{labelformat=empty}
    \caption{\footnotesize The EER of min, mean and median criteria over \n}
    \label{tab:parameters condition}
    \input{Manuscripts/src/tables/\n-Model-Type-EER}
    \end{table}
    
    \begin{block}{\footnotesize Conditions}
        \tiny These results are average over the results of below conditions: correlation and distant score, All PCs and 95\% variances and Min-max and z-score algorithm. also all features in \n \ were considered.
    \end{block}
    
    \end{frame}
    
    
    
    \begin{frame}
    \centering
    \frametitle{Comparing different criteria over \n}
    \includegraphics[scale=0.3]{Manuscripts/src/figures/\n-Model-Type.png}
    \end{frame}
    
    }
\section{Features}
    \foreach \n in {pfeatures}{%{afeatures-simple, afeatures-otsu, pfeatures, COX-time-series}{
    \begin{frame}
    \frametitle{Comparing different extracted features over \n}
    \tiny
    \begin{table}
    \centering
    \captionsetup{labelformat=empty}
    \caption{\footnotesize The accuracy of different extracted features over \n}
    \input{Manuscripts/src/tables/\n-Acc}
    \end{table}
    
    
    \begin{block}{\footnotesize Conditions}
        \tiny These results are average over the results of below conditions: min, mean and median criteria, correlation and distant score, All PCs and 95\% variances and Min-max and z-score algorithm.
    \end{block}
    
    \end{frame}
    
    \begin{frame}
    \frametitle{Comparing different extracted features over \n}
    \tiny
    \begin{table}
    \centering
    \captionsetup{labelformat=empty}
    \caption{\footnotesize The EER of different extracted features over \n}
    \label{tab:parameters condition}
    \input{Manuscripts/src/tables/\n-EER}
    \end{table}
    
    \begin{block}{\footnotesize Conditions}
        \tiny These results are average over the results of below conditions: min, mean and median criteria, correlation and distant score, All PCs and 95\% variances and Min-max and z-score algorithm.
    \end{block}
    
    \end{frame}
    
    \begin{frame}
    \centering
    \frametitle{Comparing different extracted features over \n}
    \includegraphics[scale=0.3]{Manuscripts/src/figures/\n.png}
    \end{frame}
    
    }
\section{Features selection}
    \foreach \n in {pfeatures}{%{afeatures-simple, afeatures-otsu, pfeatures}{
    \begin{frame}
    \frametitle{Comparing different extracted features over \n \ based on FS algorithm}
    \tiny
    \begin{table}
    \centering
    \captionsetup{labelformat=empty}
    \caption{\footnotesize The top 10 features over \n}
    \input{Manuscripts/src/tables/\n-10best-FS}
    \end{table}
    
    
  
    
    \end{frame}
    
    \begin{frame}
    \frametitle{Comparing different extracted features over \n \ based on FS algorithm}
    \tiny
    \begin{table}
    \centering
    \captionsetup{labelformat=empty}
    \caption{\footnotesize The top 10 features over \n}
    \label{tab:parameters condition}
    \input{Manuscripts/src/tables/\n-10worst-FS}
    \end{table}
    
    \end{frame}
    

    
    }
\section{The best and worst top models}
    \begin{frame}[shrink = 35]
    \frametitle{The best top models}
    \tiny
    \begin{table}
    \centering
    \captionsetup{labelformat=empty}
    \caption{\footnotesize The top 10 models on left foot.}
    \input{Manuscripts/src/tables/top10-L}
    \end{table}
    
    
    \begin{table}
    \centering
    \captionsetup{labelformat=empty}
    \caption{\footnotesize The top 10 models on right foot.}
    \input{Manuscripts/src/tables/top10-R}
    \end{table}
    
    
    \end{frame}
    

    
    
    
    \begin{frame}[shrink = 35]
    \frametitle{The worst top models}
    \tiny
    \begin{table}
    \centering
    \captionsetup{labelformat=empty}
    \caption{\footnotesize The worst 10 models on left foot.}
    \input{Manuscripts/src/tables/worse10-L}
    \end{table}
    
    
    \begin{table}
    \centering
    \captionsetup{labelformat=empty}
    \caption{\footnotesize The worst 10 models on right foot.}
    \input{Manuscripts/src/tables/worse10-R}
    \end{table}
    
    \end{frame}



\end{document}

