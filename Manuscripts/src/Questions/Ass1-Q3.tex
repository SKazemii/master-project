\item \textbf{Try modeling the residuals as an AR process. Use the tools at your disposal to decide on an appropriate order and analyse the results. What is the impact of selecting different orders on the remaining residuals?}




\textit{At the starting point, we need to convert our data to a stationary data. For this reason, we used the first-order differencing at the second dataset (see part \ref{Q3} for more information).}
%so these are my starting points
%must be stationary
%these two plot do not tell us much
\textit{Afterward, we need to select the order of the \gls{AR} term (p). Therefore, \gls{PACF} and \gls{ACF} were plotted at the second step (see figures \ref{fig:Ass1_D1_PACF_ACF_X} and \ref{fig:Ass1_D2_PACF_ACF_X}).}  

\begin{figure}[H]
    \centering
    \begin{minipage}[b]{1\textwidth}
        \includegraphics[width=\textwidth]{figures/Ass1/Ass1_D1_PACF_ACF_X.png}
    \end{minipage}
    \caption{A plot of the \gls{PACF} and \gls{ACF} on the residual part of the first dataset (residual of the STL method).}
    \label{fig:Ass1_D1_PACF_ACF_X}
\end{figure}

\begin{figure}[H]
    \centering
    \begin{minipage}[b]{1\textwidth}
        \includegraphics[width=\textwidth]{figures/Ass1/Ass1_D2_PACF_ACF_X.png}
    \end{minipage}
    \caption{A plot of the \gls{PACF} and \gls{ACF} on the $1^{st}$ order differencing in the second dataset.}
    \label{fig:Ass1_D2_PACF_ACF_X}
\end{figure}


\textit{Since \gls{ACF} is decaying exponentially in both figures, it can conclude that these processes are an Auto-Regressive process. Also, based on \gls{PACF} in the first plot (figure \ref{fig:Ass1_D1_PACF_ACF_X}), the order of the \gls{AR} model should be either 1 and 2 due to these two lags have a significant value. Likewise, for the second dataset, we should select p a value between 1 to 4 (see figure \ref{fig:Ass1_D2_PACF_ACF_X}).}

\begin{figure}[H]
    \centering
    \begin{minipage}[b]{1\textwidth}
        \includegraphics[width=\textwidth]{figures/Ass1/Ass1_D1_ARs models.png}
    \end{minipage}
    \caption{A part of residual signal (blue) and the fitted values (orange) for the first dataset.}
    \label{fig:Ass1_D1_ARs_models}
\end{figure}

\begin{figure}[H]
    \centering
    \begin{minipage}[b]{1\textwidth}
        \includegraphics[width=\textwidth]{figures/Ass1/Ass1_D2_ARs models.png}
    \end{minipage}
    \caption{A part of residual signal (blue) and the fitted values (orange) for the second dataset.}
    \label{fig:Ass1_D2_ARs_models}
\end{figure}


\textit{Figures \ref{fig:Ass1_D1_ARs_models} and \ref{fig:Ass1_D2_ARs_models} indicate the output of our models on the datasets. As these two figures indicate, it is hard to select the best model based on the trained signal. Therefore, we need some criteria to help us to find the best model. Tables \ref{tab:Ass1_D1_AR} and \ref{tab:Ass1_D2_AR} show these criteria on these models. }

\begin{table}[H]
\centering
\caption{Comparing the \gls{AR} models in the first dataset.}
\label{tab:Ass1_D1_AR}
\input{tables/Ass1/Ass1_D1_AR.tex}
\end{table}

\begin{table}[H]
\centering
\caption{Comparing the \gls{AR} models in the second dataset.}
\label{tab:Ass1_D2_AR}
\input{tables/Ass1/Ass1_D2_AR.tex}
\end{table}

\textit{Figures \ref{fig:Ass1_D1_Training Errors of ARs models} and \ref{fig:Ass1_D2_Training Errors of ARs models} indicate the Training Errors of \gls{AR} models on the training set. As you can see, the training error are residual.}


\begin{figure}[H]
    \centering
    \begin{minipage}[b]{1\textwidth}
        \includegraphics[width=\textwidth]{figures/Ass1/Ass1_D1_Training Errors of ARs models.png}
    \end{minipage}
    \caption{The Training Errors of AR models for the first dataset. }
    \label{fig:Ass1_D1_Training Errors of ARs models}
\end{figure}

\begin{figure}[H]
    \centering
    \begin{minipage}[b]{1\textwidth}
        \includegraphics[width=\textwidth]{figures/Ass1/Ass1_D2_Training Errors of ARs models.png}
    \end{minipage}
    \caption{The Training Errors of AR models for the second dataset. }
    \label{fig:Ass1_D2_Training Errors of ARs models}
\end{figure}

\textit{\Gls{AIC} and \gls{BIC} show the simplicity and goodness of an \gls{AR} model. If a model has a lower \Gls{AIC} and \gls{BIC}, it will be generally better than others. So for the first dataset, \gls{AR}(2) is better than \gls{AR}(1), and for the second dataset, \gls{AR}(4) is lower than among others.Figures \ref{fig:Ass1_D1_AR} and \ref{fig:Ass1_D2_AR} show the result of the model on testing set.}




\begin{figure}[H]
    \centering
    \begin{minipage}[b]{1\textwidth}
        \includegraphics[width=\textwidth]{figures/Ass1/Ass1_D1_AR.png}
    \end{minipage}
    \caption{The prediction and holding out set (test set) of the first dataset. }
    \label{fig:Ass1_D1_AR}
\end{figure}

\begin{figure}[H]
    \centering
    \begin{minipage}[b]{1\textwidth}
        \includegraphics[width=\textwidth]{figures/Ass1/Ass1_D2_AR.png}
    \end{minipage}
    \caption{The prediction and holding out set (test set) of the second dataset. }
    \label{fig:Ass1_D2_AR}
\end{figure}